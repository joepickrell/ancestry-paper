
\documentclass[11pt,titlepage]{article} 
\usepackage[square,sort&compress]{natbib}
%\usepackage[round]{natbib} 
\usepackage{amsfonts}
\usepackage{amsmath}
\usepackage{amssymb}
\usepackage{hyperref}
\usepackage{longtable}

\usepackage{caption}
\captionsetup{labelsep = none}
\usepackage{graphicx,rotating}

\usepackage{setspace}
\usepackage[table]{xcolor}
\definecolor{lightgray}{gray}{0.9}
%\bibliographystyle{elsart-harv}

%\bibliographystyle{JRSS}
\bibliographystyle{genres}
%\usepackage[margin=1in]{geometry}
\newcommand{\mb}[1]{\mathbf{#1}}
\newcommand{\wt}[1]{\widetilde{#1}}
\doublespacing
\renewcommand{\rmdefault}{ptm} 
\usepackage{mathptmx} 
\usepackage{tikz}
\usetikzlibrary{arrows}
\usepackage[margin=1in]{geometry}
\renewcommand{\baselinestretch}{1.05}    %or 1.4 maybe?
%\setlength{\oddsidemargin}{0cm} \setlength{\textwidth}{6.5in}
%\setlength{\topmargin}{0in} \setlength{\topmargin}{-.5in}

\newcolumntype{C}[1]{>{\centering\let\newline\\\arraybackslash\hspace{0pt}}m{#1}}
\newcommand\encircle[1]{%
  \tikz[baseline=(X.base)] 
    \node (X) [draw, shape=circle, inner sep=0] {\strut #1};}
%this version seems to work better for pdf (?)
\setlength{\textheight}{9.5in}
%%
%% Select one of the two - the second gets rid of comments.
%% Leave the blank lines in
\def\comment#1{

\smallskip\noindent{\it [{#1}]}

\smallskip\noindent
}
%\def\comment#1{}


\begin{document}
\title{Efficient ancestry inference for personal genomics}
\author{Pickrell, Gordon, Berisa, Erlich\\ \\
\small $^1$ New York Genome Center, New York, NY, USA\\
\small $^2$ Department of Biological Sciences, Columbia University, New York, NY, USA \\
\small $^2$ Department of Computer Science, Columbia University, New York, NY, USA \\
\small $^\dagger$ Correspondence to: \url{jkpickrell@nygenome.org}
}
\maketitle
\begin{abstract}
We describe an implementation of the Pritchard-Stephens-Donnelly ``STRUCTURE" model that is designed for estimation of an individual's ancestry from single nucleotide polymorphism (SNP) data. In our implementation, we assume we have a reference panel of individuals of known ancestry, and then model a test individual (of unknown ancestry) as a mixture of these reference populations. We describe three applications. First, we applied the method to the 150,000 individuals genotyped in the UK Biobank to select individuals for inclusion in an association study; second, we applied the method to consumer genomics data from 15,000 participants DNA.Land; and finally we applied it to estimate the ancestry of XX individuals whose tissues were used in ENCODE/Roadmap Epigenomics projects. Software (released under a GPL license) and a reference panel are available at \url{https://bitbucket.org/joepickrell/ancestry}
\end{abstract}
%\tableofcontents
\clearpage
\section{Introduction}
Estimation of an individual's ``genetic ancestry" is useful in a number of contexts. First, in the context of genetic association studies. 

There are a large number of methods and/or software packages that can be used for ancestry inference (lots of citations...). Our initial goal was to identify a piece of software that could be used in ``production" (that is, with minimal failure rates and no need for manual intervention) on many SNP datasets as part of DNA.Land and related efforts. We thus primarily considered algorithms that have been proven to work in this context; i.e. those used by popular commercial ancestry testing companies. The company 23andMe attempts to perform ``local" ancestry inference using a custom method \cite{durand2014ancestry}, while AncestryDNA and Family Tree DNA perform ``global" ancestry inference using the software package ADMIXTURE \citep{Alexander:2009fk}, which is an implementation of a model first developed by \citet{Pritchard:2000zr}. Since we were primarily interested in estimating global ancestry, we focused on the latter. 

\section{Results}
We use a supervised version of the ``admixture" model from \citet{Pritchard:2000zr}. 
Specifically, assume we have an individual of unknown ancestry (the ``test individual") and a panel of individuals with labeled ancestry who come from a total of $M$ populations (the ``reference panel"). 
We assume the individual and the reference panel have been genotyped at $S$ bi-allelic single nucleotide polymorphisms (SNPs).
 At each SNP, we arbitrarily define one allele as the ``reference" and the other as the ``alternate". 
 
 Let $\vec a = {a_1, a_2, .., a_S}$ be a vector of indicator variables that each have value 1 if the test individual has the reference allele at position $i$ and zero otherwise (assume the test individual is haploid, such that there is only a single allele to evaluate). 
Let $C$ be the matrix of counts of a (arbitrarily-chosen) reference allele at locus $i$ in individuals labeled as coming from population $j$ (such that $C$ has $S$ rows and $M$ columns), and let $N$ be the matrix of total number of alleles sampled from population $j$ at each locus. 
Let $p_{ij}$ be the allele frequency of the reference allele at locus $i$ in population $j$, and $\vec p_i$ be the vector of the allele frequencies in all reference populations at locus $i$. 
 Finally, let $\vec q$ be the vector containing the proportion of ancestry that the test individual has from each of the reference populations, such that $\sum_i q_i = 1$. 

Now we can write down the likelihood at a single SNP:

\begin{equation}
L(\vec p_i, \vec q | \vec c_i, \vec n_i, a_i) = a_i f_i \prod  \limits_{j = 1}^M {n_{ij} \choose c_{ij}} p_{ij}^{c_{ij}} (1-p_{ij})^{n_{ij}- c_{ij}} , 
\end{equation}
\noindent where $f_i = \sum \limits_{j = 1}^{M} p_{ij} q_j$.  

We treat all SNPs as independent, so the complete likelihood function is:

\begin{equation}
L(P, \vec q | C,  N, \vec a) = \prod \limits_{i = 1}^{S} L(\vec p_i, \vec q | \vec c_i, \vec n_i, a_i).
\end{equation}

\subsection{Estimation}
We now describe our approach to getting the maximum likelihood estimates of $P$ and $\vec q$. 

\section{Discussion}

\section{Methods}
\bibliography{bib}
\end{document}





