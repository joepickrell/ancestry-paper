
\documentclass[11pt,titlepage]{article} 
\usepackage[square,sort&compress]{natbib}
%\usepackage[round]{natbib} 
\usepackage{amsfonts}
\usepackage{amsmath}
\usepackage{amssymb}
\usepackage{hyperref}
\usepackage{longtable}

\usepackage{caption}
\captionsetup{labelsep = none}
\usepackage{graphicx,rotating}

\usepackage{setspace}
\usepackage[table]{xcolor}
\definecolor{lightgray}{gray}{0.9}
%\bibliographystyle{elsart-harv}

%\bibliographystyle{JRSS}
\bibliographystyle{genres}
%\usepackage[margin=1in]{geometry}
\newcommand{\mb}[1]{\mathbf{#1}}
\newcommand{\wt}[1]{\widetilde{#1}}
\doublespacing
\renewcommand{\rmdefault}{ptm} 
\usepackage{mathptmx} 
\usepackage{tikz}
\usetikzlibrary{arrows}
\usepackage[margin=1in]{geometry}
\renewcommand{\baselinestretch}{1.05}    %or 1.4 maybe?
%\setlength{\oddsidemargin}{0cm} \setlength{\textwidth}{6.5in}
%\setlength{\topmargin}{0in} \setlength{\topmargin}{-.5in}

\newcolumntype{C}[1]{>{\centering\let\newline\\\arraybackslash\hspace{0pt}}m{#1}}
\newcommand\encircle[1]{%
  \tikz[baseline=(X.base)] 
    \node (X) [draw, shape=circle, inner sep=0] {\strut #1};}
%this version seems to work better for pdf (?)
\setlength{\textheight}{9.5in}
%%
%% Select one of the two - the second gets rid of comments.
%% Leave the blank lines in
\def\comment#1{

\smallskip\noindent{\it [{#1}]}

\smallskip\noindent
}
%\def\comment#1{}


\begin{document}
\title{Efficient ancestry inference for personal genomics}
\author{Pickrell, Gordon, Berisa, Erlich\\ \\
\small $^1$ New York Genome Center, New York, NY, USA\\
\small $^2$ Department of Biological Sciences, Columbia University, New York, NY, USA \\
\small $^2$ Department of Computer Science, Columbia University, New York, NY, USA \\
\small $^\dagger$ Correspondence to: \url{jkpickrell@nygenome.org}
}
\maketitle
\begin{abstract}
Estimation of individual ancestry is important. Software and a reference panel are available at \url{https://bitbucket.org/joepickrell/ancestry}
\end{abstract}
%\tableofcontents
\clearpage
\section{Introduction}
Estimation of an individual's ``genetic ancestry" is useful in a number of contexts. First, in the context of genetic association studies, 

\section{Results}
We use a supervised version of the ``admixture" model from \citet{Pritchard:2000zr}. Specifically, assume we have an individual of unknown ancestry (the ``test individual") and a panel of individuals with labeled ancestry who come from a total of $N$ populations (the ``reference panel"). We assume the individual and the reference panel have been genotyped at $S$ bi-allelic single nucleotide polymorphisms (SNPs). At each SNP, we arbitrarily define one allele as the ``reference" and the other as the ``alternate". Let $a_i$ be an indicator variable with value 1 if the test individual has the reference allele at position $i$ and zero otherwise (assume the test individual is haploid, such that there is only a single allele to evaluate). Let $c_{ij}$ be the count of a (arbitrarily-chosen) reference allele at locus $i$ in individuals labeled as coming from population $j$, and let $n_{ij}$ be total number of alleles sampled from population $j$ at the locus. Let $p_{ij}$ be the allele frequency of the reference allele at locus $i$ in population $j$, and $\vec p_i$ be the vector of the allele frequencies in all reference populations. Finally, let $q_j$ be the fraction of ancestry in the test individual from each of the reference populations, such that $\sum_j q_j = 1$. 

Now we can write down the likelihood of the data at a single SNP:

\begin{equation}
L(\vec p_i, \vec q | \vec c_i, \vec n_i, a_i) = a_i f_i \prod  \limits_{j = 1}^N {n_{ij} \choose c_{ij}} p_{ij}^{c_{ij}} (1-p_{ij})^{n_{ij}- c_{ij}} , 
\end{equation}
\noindent where $f_i = \sum \limits_{j = 1}^{N} p_{ij} q_j$.  

\subsection{Estimation}
We now describe our approach to getting the maximum likelihood estimates of $\vec p$ and $Q$. 

\section{Discussion}

\section{Methods}
\bibliography{bib}
\end{document}





